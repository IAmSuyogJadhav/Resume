%%%%%%%%%%%%%%%%%%%%%%%%%%%%%%%%%%%%%%%
% Deedy - One Page Two Column Resume
% LaTeX Template
% Version 1.2 (16/9/2014)
%
% Original author:
% Debarghya Das (http://debarghyadas.com)
%
% Original repository:
% https://github.com/deedydas/Deedy-Resume
%
% IMPORTANT: THIS TEMPLATE NEEDS TO BE COMPILED WITH XeLaTeX
%
% This template uses several fonts not included with Windows/Linux by
% default. If you get compilation errors saying a font is missing, find the line
% on which the font is used and either change it to a font included with your
% operating system or comment the line out to use the default font.
% 
%%%%%%%%%%%%%%%%%%%%%%%%%%%%%%%%%%%%%%
% 
% TODO:
% 1. Integrate biber/bibtex for article citation under publications.
% 2. Figure out a smoother way for the document to flow onto the next page.
% 3. Add styling information for a "Projects/Hacks" section.
% 4. Add location/address information
% 5. Merge OpenFont and MacFonts as a single sty with options.
% 
%%%%%%%%%%%%%%%%%%%%%%%%%%%%%%%%%%%%%%
%
% CHANGELOG:
% v1.1:
% 1. Fixed several compilation bugs with \renewcommand
% 2. Got Open-source fonts (Windows/Linux support)
% 3. Added Last Updated
% 4. Move Title styling into .sty
% 5. Commented .sty file.
%
%%%%%%%%%%%%%%%%%%%%%%%%%%%%%%%%%%%%%%%
%
% Known Issues:
% 1. Overflows onto second page if any column's contents are more than the
% vertical limit
% 2. Hacky space on the first bullet point on the second column.
%
%%%%%%%%%%%%%%%%%%%%%%%%%%%%%%%%%%%%%%


\documentclass[]{deedy-resume-openfont}
\usepackage{fancyhdr}
 
\pagestyle{fancy}
\fancyhf{}
\fancyhead[LE,RO]{Overleaf}
\fancyhead[RE,LO]{Guides and tutorials}
\fancyfoot[CE,CO]{\leftmark}
\fancyfoot[LE,RO]{\thepage}

 
\begin{document}

%%%%%%%%%%%%%%%%%%%%%%%%%%%%%%%%%%%%%%
%
%     LAST UPDATED DATE
%
%%%%%%%%%%%%%%%%%%%%%%%%%%%%%%%%%%%%%%
\lastupdated


%%%%%%%%%%%%%%%%%%%%%%%%%%%%%%%%%%%%%%
%
%     TITLE NAME
%
%%%%%%%%%%%%%%%%%%%%%%%%%%%%%%%%%%%%%%
\namesection{Suyog}{Jadhav}

%%%%%%%%%%%%%%%%%%%%%%%%%%%%%%%%%%%%%%
%
%     COLUMN ONE
%
%%%%%%%%%%%%%%%%%%%%%%%%%%%%%%%%%%%%%%

\begin{minipage}[t]{0.28\textwidth} 

%%%%%%%%%%%%%%%%%%%%%%%%%%%%%%%%%%%%%%
%     EDUCATION
%%%%%%%%%%%%%%%%%%%%%%%%%%%%%%%%%%%%%%

\section{Education} 

\subsection{IIT (ISM), Dhanbad}
\descript{BTech in Electronics and Communication}
\location{May 2021 | Dhanbad (JH), India}
\location{ Cum. GPA: 8.205 / 10.0
% \\ Major GPA: -- / 10.0
}
\sectionsep

\subsection{Modern College, Pune}
\location{Grad. May 2017 |  Pune (MH), India}
\sectionsep

%%%%%%%%%%%%%%%%%%%%%%%%%%%%%%%%%%%%%%
%     LINKS
%%%%%%%%%%%%%%%%%%%%%%%%%%%%%%%%%%%%%%

\section{Links} 
\href{mailto:suyog.jadhav1@gmail.com}{\bf Mail} \\
\href{http://suyogjadhav.com}{\bf Website} \\
\href{https://scholar.google.com/citations?user=8wS71BQAAAAJ&hl=en}{\bf Google Scholar Profile} \\
\href{https://www.suyogjadhav.com/resume/}{\textbf{Complete Unabridged Resume}}

\sectionsep
Github:// \href{https://github.com/IAmSuyogJadhav}{\bf IAmSuyogJadhav} \\
LinkedIn://  \href{https://www.linkedin.com/in/IAmSuyogJadhav}{\bf IAmSuyogJadhav} \\
Twitter://  \href{https://twitter.com/IAmSuyogJadhav}{\bf IAmSuyogJadhav} \\

%%%%%%%%%%%%%%%%%%%%%%%%%%%%%%%%%%%%%%
%     COURSEWORK
%%%%%%%%%%%%%%%%%%%%%%%%%%%%%%%%%%%%%%

% \section{Coursework}
% \subsection{Undergraduate}
% Information Retrieval \\
% Operating Systems \\
% Artificial Intelligence + Practicum \\
% Functional Programming \\
% Computer Graphics + Practicum \\
% {\footnotesize \textit{\textbf{(Research Asst. \& Teaching Asst 2x) }}} \\
% Unix Tools and Scripting \\

%%%%%%%%%%%%%%%%%%%%%%%%%%%%%%%%%%%%%%
%     SKILLS
%%%%%%%%%%%%%%%%%%%%%%%%%%%%%%%%%%%%%%
\sectionsep
\section{Skills}
\subsection{Programming}
\location{Very Familiar:}
Python \textbullet{} PyTorch \textbullet{} API and backend development (Flask) \textbullet{} Linux \textbullet{} Git \textbullet{} Google Cloud Platform

\location{Over 1000 lines of code:}
C \textbullet{} C++ \textbullet{}  Matlab \textbullet{} Regex \textbullet{} CSS \textbullet{} Keras \textbullet{} TensorFlow

\location{Familiar:}
Shell \textbullet{} Javascript \textbullet{} Batch

\sectionsep

%%%%%%%%%%%%%%%%%%%%%%%%%%%%%%%%%%%%%%
%     AWARDS
%%%%%%%%%%%%%%%%%%%%%%%%%%%%%%%%%%%%%%

\section{Awards/Scores} 

\descript{Gold Medal | Dec 2019}
\location{Ashoka's Tech for Change Challenge at 6\textsuperscript{th} Inter-IIT Tech Meet}
\sectionsep
\descript{2\textsuperscript{nd}/150 teams | Sep 2019}
\location{CDAC AI Hackathon 2019 sponsored by Nvidia}
\sectionsep
\descript{2\textsuperscript{nd} Rank | Apr 2020}
\location{EndoCV Challenge (Single Frame Object Detection track), 17th IEEE ISBI (2020) [2]}
\sectionsep
\descript{4\textsuperscript{th}/300 teams | Jan 2019}
\location{PanIIT Mission AI: Solve for India Hackathon}
\sectionsep
\descript{IELTS Band 8/9 | Oct 2020}
\location{CEFR Level: C1}


% \begin{tabular}{rll}
% 2019	     & Gold Medal  & Ashoka's Tech for Change Challenge \\ & & at Inter-IIT Tech Meet 2019 \\
% 2019         & 2\textsuperscript{nd}/150 teams & CDAC AI Hackathon 2019 sponsored by Nvidia \\
% 2019         & 4\textsuperscript{th}/300 teams & PanIIT Mission AI: Solve for India Hackathon

% \end{tabular}
\sectionsep

%%%%%%%%%%%%%%%%%%%%%%%%%%%%%%%%%%%%%%
%
%     COLUMN TWO
%
%%%%%%%%%%%%%%%%%%%%%%%%%%%%%%%%%%%%%%

\end{minipage} 
\hfill
\begin{minipage}[t]{0.7\textwidth} 

%%%%%%%%%%%%%%%%%%%%%%%%%%%%%%%%%%%%%%
%     EXPERIENCE
%%%%%%%%%%%%%%%%%%%%%%%%%%%%%%%%%%%%%%
\section{Experience}

\runsubsection{UiT- The Arctic University of Norway}
\descript{| Research Intern }
\location{Apr 2020 - Aug 2020 | Tromsø, Norway}
% \vspace{\topsep} % Hacky fix for awkward extra vertical space
\location{Worked on two different projects during the course of this internship.}
\vspace{\topsep} % Hacky fix for awkward extra vertical space
\begin{tightemize}
\item Application of deep learning for illumination estimation in Fourier ptychography microscopy (FPM);
\item Artefact removal from MUSICAL nanoscopy images using deep learning.
\end{tightemize}
% Both the projects have been discussed in more detail under the \textit{Research} section. 
\sectionsep

\runsubsection{Cancer Moonshot Inc.}
\descript{| Deep Learning R\&D Intern }
\location{Jun 2019 – Jul 2019 | Bangalore (KA), India}
\location{Worked on developing a deep learning model for segmentation of cancer lesions from prostate MRI scans.}
% \vspace{\topsep} % Hacky fix for awkward extra vertical space
\begin{tightemize}
\item Trained UNet model with custom modifications to fit the extremely small ground truth labels. Used data augmentation techniques to improve the performance of the model. 
\end{tightemize}
\sectionsep

% \runsubsection{DataProrrisi Inc.}
% \descript{| Software Engineering Intern }
% \location{Dec 2018 – Jan 2019 | Remote}
% \begin{tightemize}
% \item Assisted in developing the backend for DataProrrisi, a startup focused on revolutionizing the loan acquisition process using machine learning, based in California.
% \item I used Flask for implementing the necessary API calls needed by the backend.
% \end{tightemize}
% \sectionsep

% \runsubsection{Cyber Labs - IIT (ISM) Dhanbad} %- The Arctic University of Norway}
% \descript{| AI Team Leader}
% \location{Jul 2019 - Present | Dhanbad (JH), India}
% \begin{tightemize}
% \item My team works on various projects that utilise machine learning. We also conduct workshops, hold paper reading sessions and organize machine learning competitions in the college.
% \end{tightemize}
% \sectionsep

%%%%%%%%%%%%%%%%%%%%%%%%%%%%%%%%%%%%%%
%     RESEARCH
%%%%%%%%%%%%%%%%%%%%%%%%%%%%%%%%%%%%%%

\section{Past Research Projects}

\runsubsection{Artefact Removal from Nanoscopy Images}
% \descript{ }
\descript{| Aug 2020}
MUSICAL is a nanoscopy method that produces a high-res output from a temporal stack of fluorescence microscopy images. The produced MUSICAL image has artefacts due to input noise. I worked on simulating 3 different subcellular structures and training autoencoder models for denoising the produced MUSICAL images. \emph{Accepted in Biomedical Optics Express (Dec. 2020)} [1].
\sectionsep

\runsubsection{Deep Learning in Fourier Ptychography}
% \descript{ }
\descript{| Jun 2020}
Fourier Ptychography is a microscopy technique that uses low-res images taken from multiple angles to generate a high-res image. I implemented the complete pipeline including the object detection based illumination angle estimation model, calibration and the final reconstruction algorithm adapted from Aidukas et. al. 2018. \emph{Published in Optics Express Journal (Dec. 2020)} [3].
\sectionsep

\runsubsection{ChestX}
% \descript{ }
\descript{| Sep 2019}
Developed a computer-aided diagnosis system for detecting 14 chest abnormalities from X-ray scans. Used a novel 3-stage deep learning architecture and achieved a maximum AUC score of 0.91 (on Emphysema) with the average AUC score being 0.84. The system also calculates and displays the class activation maps for each of the classes, aiding in interpretation of the results. Secured 2nd rank in CDAC AI hackathon 2019 co-sponsored by Nvidia.
\sectionsep

% \runsubsection{Deep Learning in Fourier Ptychography}
% \descript{ }
\descript{$\mapsto$ Details about more projects can be found on  \href{https://github.com/IAmSuyogJadhav}{\bf my GitHub profile}}
\sectionsep


% \runsubsection{Cornell Phonetics Lab}
% \descript{| Head Undergraduate Researcher}
% \location{Mar 2012 – May 2013 | Ithaca, NY}
% Led the development of \textbf{QuickTongue}, the first ever breakthrough tongue-controlled game with \textbf{\href{http://conf.ling.cornell.edu/~tilsen/}{Prof Sam Tilsen}} to aid in Linguistics research. 
\sectionsep

%%%%%%%%%%%%%%%%%%%%%%%%%%%%%%%%%%%%%%
%     PUBLICATIONS
%%%%%%%%%%%%%%%%%%%%%%%%%%%%%%%%%%%%%%

\section{Publications} 
\renewcommand\refname{\vskip -1.5em} % Couldn't get this working from the .cls file
\bibliographystyle{abbrv}
\bibliography{publications}
\nocite{*}

\end{minipage} 
\end{document}  \documentclass[]{article}

% @article{jadhav2020talk,
%   title={invited talk on "Machine Learning Based Analytics from Wearable Sensors"},
%   author={Suyog Jadhav and Dilip K. Prasad},
%   journal="workshop on Machine Learning and its Application in Sport Science and Public Health, TU Munich, Munich, Germany",
%   month=Feb,
%   year={2020}
% }
